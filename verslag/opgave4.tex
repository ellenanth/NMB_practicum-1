Neem een vector $x \in \mathbb{R}^n$. Schrijf $x$ als lineaire combinatie van de eigenvectoren van $A$ $q_1, q_2 \dots q_n$ met bijhorende eigenwaarden $\lambda_1, \lambda_2 \dots \lambda_n$:
$$ x = \sum_{j=1}^{n}a_jq_j,$$
dan is het Rayleigh quoti\"ent van $x$:
$$r(x) = \frac{\sum_{j=1}^{n}a_j^2q_j\lambda_j}{\sum_{j=1}^{n}a_j^2}.$$
Het Rayleigh quoti\"ent is onafhankelijk van de schaal van $x$, dus stel $\lVert a \rVert = \lVert \begin{bsmallmatrix} a_1&a_2&\dots&a_n\end{bsmallmatrix}^T \rVert = 1$, dan is $\sum_{j=1}^{n}a_j^2 = 1$. Dan wordt
$$r(x) = \sum_{j=1}^{n}a_j^2q_j\lambda_j.$$
Stel nu dat $\lambda_1 \geq \lambda_2 \geq \dots \geq \lambda_n$, dan is het Rayleigh quoti\"ent maxiaal voor $a = e_1$ met de waarde $\lambda_{max}$ en minimaal voor $a = e_n$ met de waarde $\lambda_{min}$. Dus het Rayleigh quoti\"ent bevint zich in het interval $\begin{bsmallmatrix}\lambda_{min},\lambda_{max}\end{bsmallmatrix}$.

Ook is het Rayleigh queti\"ent een continu voor $a \neq 0$, dus elke waarde tussen $\lambda_{min}$ en $\lambda_{max}$ wordt bereikt voor een $x$.