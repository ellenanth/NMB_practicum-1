Neme een vector $x \in \mathbb{R}^n$. Schrijf $x$ als lineaire combinatie van de eigenvectoren $q_1, q_2 \dots q_n$ met bijhorende eigenwaarden $\lambda_1, \lambda_2 \dots \lambda_n$ van $A$:
$$ x = \sum_{j=1}^{n}a_jq_j,$$
dan is het Rayleigh quoti\"ent van $x$:
$$r(x) = \frac{\sum_{j=1}^{n}a_j^2q_j\lambda_j}{\sum_{j=1}^{n}a_j^2}.$$
Het Rayleigh quoti\"ent is onafhankelijk van de schaal van $x$, dus stel $\lVert x \rVert = 1$, dan is $\sum_{j=1}^{n}a_j^2 = 1$. Dan wordt
$$r(x) = \sum_{j=1}^{n}a_j^2q_j\lambda_j.$$
Deze is maxi
