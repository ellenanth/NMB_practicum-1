Een algoritme in pseudo-code dat alle eigenwaarden van een symmetrische, tridiagonale en re\"ele matrix in het interval [a,b) berekent tot op een bepaalde tolerantie met behulp van de bisectie-methode, is te vinden in algoritme \ref{bisectie}.

\begin{algorithm}
\centering
\caption{Bisectie-methode}
    \label{bisectie}
\begin{algorithmic}
\State{queue = [[a,b)]}
\While{queue niet leeg}
\State{Haal eerste interval uit de queue}
\State{$Sturm_{links}$ = aantal tekenwisselingen in de Sturm-rij van A-linkergrens*I}
\State{$Sturm_{rechts}$ = aantal tekenwisselingen in de Sturm-rij van A-rechtergrens*I}
\State{\#eigenwaarden = $Sturm_{rechts}$ - $Sturm_{links}$}
\State{midden = 0.5*(rechtergrens - linkergrens)}

\If {\#eigenwaarden == 1}
	\If{midden $<$ tolerantie}
		\State{Voeg linkergrens+midden toe als eigenwaarde}
	\Else
    \State {Voeg [linkergrens,linkergrens+midden] en [linkergrens+midden,rechtergrens] vooraan bij in de queue}
    \EndIf
\ElsIf {\#eigenwaarden $>$ 1}
        \State {Voeg [linkergrens,linkergrens+midden] en [linkergrens+midden,rechtergrens] vooraan bij in de queue}
\EndIf
\EndWhile
\end{algorithmic}
\end{algorithm}

Deze methode werd uitgeschreven in MATLAB (zie bijgevoegde MATLAB-code) en werd toegepast op enkele symmetrische, tridiagonale en re\"ele matrices.

Als eerste voorbeeld nemen we de tridiagonale, symmetrische en re\"ele matrix $A \in \mathbb{R}^{n \times n}$ met $n=4$.
$$A = \begin{bmatrix} 
1 &5 &0 &0 \\
5 &2 &6 &0 \\
0 &6 &3 &7 \\
0 &0 &7 &4 \\
\end{bmatrix}$$
Als we de eigenwaarden van deze matrix willen berekenen in het interval [-2,6) tot op een tolerantie $10^{-2}$, dan vinden we we eigenwaarden [-1,3359375; 5,7734375]. Deze komen overeen met de 'exacte' eigenwaarden [-1,32970777874292; 5,77851991186833] berekend met de functie eig($A$) van MATLAB. Als we beide resultaten afronden tot op twee cijfers na de komma, zien we inderdaad dat de eigenwaarden gevonden zijn tot op de gegeven absolute fout.

Als tweede voorbeeld genereren we de random symmetrische, tridiagonale en re\"ele matrix $B \in \mathbb{R}^{n \times n}$ MATLAB met $n = 10$. De diagonalen zijn berekend met de functie rand($n$) en rand($n-1$). De elementen van de matrix $B$ zijn hier afegrond tot op 4 cijfers na de komma.
 $$B = \begin{bmatrix}
0,4170	&0,2068	&0	&0	&0	&0	&0	&0	&0	&0 \\
0,2068	&0,9718	&0,6539	&0	&0	&0	&0	&0	&0	&0 \\
0	&0,6539	&0,9880	&0,0721	&0	&0	&0	&0	&0	&0 \\
0	&0	&0,0721	&0,8641	&0,4067	&0	&0	&0	&0	&0 \\
0	&0	&0	&0,4067	&0,3889	&0,6669	&0	&0	&0	&0 \\
0	&0	&0	&0	&0,6669	&0,4547	&0,9337	&0	&0	&0 \\
0	&0	&0	&0	&0	&0,9337	&0,2467	&0,8110	&0	&0 \\
0	&0	&0	&0	&0	&0	&0,8110 &0,7844	&0,4845	&0 \\
0	&0	&0	&0	&0	&0	&0	&0,4845	&0,8828	&0,7567 \\
0	&0	&0	&0	&0	&0	&0	&0	&0,7567	&0,9137 \\
\end{bmatrix}$$
Deze matrix heeft in totaal 10 eigenwaraden. Als we nu enkel ge\"interesseerd zijn in de eigenwaarden tussen 0 en 1 tot op minstens $10^{-10}$ nauwkeurig, vinden we met behulp van de bisecite-methode de eigenwaarden [0,155349893786479; 0,212609510694165; 0,511023909028154; 0,768950582772959]. Deze komen overeen met de 'exacte' eigenwaarden [0,155349893779548; 0,212609510637192; 0,511023909027886; 0,768950582751513] berekend met de functie eig($B$) in MATLAB. We zien dat de resultaten bekomen door de bisectie-methode inderdaad juist zijn tot op de gegeven tolerantie.