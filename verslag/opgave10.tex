Een algoritme in pseudo-code dat alle eigenwaarden van een symmetrische, tridiagonale en re\"ele matrix in het interval [a,b) berekent tot op een bepaalde tolerantie met behulp van de bisectie-methode, is te vinden in algoritme \ref{bisectie}.

\begin{algorithm}[H]
\centering
\caption{Bisectie-methode}
    \label{bisectie}
\begin{algorithmic}
\State{queue = [[a,b)]}
\While{queue niet leeg}
\State{Haal eerste interval uit de queue}
\State{$Sturm_{links}$ = \#tekenwisselingen in de Sturm-rij van A-linkergrens*I}
\State{$Sturm_{rechts}$ = \#tekenwisselingen in de Sturm-rij van A-rechtergrens*I}
\State{\#eigenwaarden = $Sturm_{rechts}$ - $Sturm_{links}$}
\State{midden = 0.5*(rechtergrens - linkergrens)}

\If {\#eigenwaarden == 1}
	\If{midden $<$ tolerantie}
		\State{Voeg linkergrens+midden toe als eigenwaarde}
	\Else
    \State {Voeg [linkergrens,linkergrens+midden] en [linkergrens+midden,rechtergrens] vooraan bij in de queue}
    \EndIf
\ElsIf {\#eigenwaarden $>$ 1}
        \State {Voeg [linkergrens,linkergrens+midden] en [linkergrens+midden,rechtergrens] vooraan bij in de queue}
\EndIf
\EndWhile
\end{algorithmic}
\end{algorithm}


Deze methode werd uitgeschreven in MATLAB (zie onderstaande code) en werd toegepast op enkele symmetrische, tridiagonale en re\"ele matrices.

\lstinputlisting[
  style      = Matlab-editor,
  basicstyle = \mlttfamily,
]{bisection.m}

Als eerste voorbeeld nemen we de tridiagonale, symmetrische en re\"ele matrix $A \in \mathbb{R}^{n \times n}$ met $n=4$.
$$A = \begin{bmatrix} 
1 &5 &0 &0 \\
5 &2 &6 &0 \\
0 &6 &3 &7 \\
0 &0 &7 &4 \\
\end{bmatrix}$$
Als we de eigenwaarden van deze matrix willen berekenen in het interval [-2,6) tot op een tolerantie $10^{-2}$, dan vinden we we eigenwaarden [-1,3359375; 5,7734375]. Deze komen overeen met de 'exacte' eigenwaarden [-1,32970777874292; 5,77851991186833] berekend met de functie eig($A$) van MATLAB. Als we beide resultaten afronden tot op twee cijfers na de komma, zien we inderdaad dat de eigenwaarden gevonden zijn tot op de gegeven absolute fout.

Een interessant testprobleem is een random symmetrische, tridiagonale en re\"ele matrix $B \in \mathbb{R}^{n \times n}$ MATLAB met $n = 100$. De diagonalen zijn berekend met de functie rand($n$) en rand($n-1$). We zoeken naar de eigenwaarden in het interval [0,1) tot op een tolerantie $10^{-2}$. De matrix waarmee dit getest is, wordt hier niet weergegeven omdat deze veel te groot is. Het interessante aan deze matrix is dat de eigenwaarden die tussen 0 en 1 gelegen zijn, meestal dichter bij elkaar liggen dan de opgegeven tolerantie. Het algoritme dat hierboven beschreven wordt, kan toch nog steeds onderscheid maken tussen de verschillende eigenwaarden. Zo vindt het bijvoorbeeld als eerste twee eigenwaarden 0,03515625 en 0,04296875 voor de eigenwaarden (berekend met de eig() functie van MATLAB) 0,0340437309188287 en 0,0441523470170203.

Een volgend interessant testprobleem wordt bekomen door eigenwaarden als grens te nemen. Theoretisch gezien mag de eigenwaarde die de rechtergrens voorstelt niet gevonden worden. Dit is echter soms toch het geval (zeker voor grote matrices). Dit komt doordat de elementen van de Sturm-rij opgesteld worden aan de hand van vorige elementen. Zo kunnen reken- en afrondingsfouten zich voortplanten.

Ook is er tijdens het testen opgevallen dat de gevraagde nauwkeurigheid niet altijd bereikt wordt voor alle eigenwaarden. Dit komt doordat wanneer de grens van een te onderzoeken interval dicht bij een eigenwaarde komt, de Sturm-rij van $A-xI$ slecht geconditioneerd is. Het laatste element ligt dan rond 0, waardoor er snel een tekenwissel te veel of te weinig is.