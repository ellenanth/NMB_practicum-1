Een algoritme in pseudo-code dat alle eigenwaarden van een symmetrische tridiagonale matrix in het interval [a,b) berekent tot op een bepaalde tolerantie met behulp van de bisectie-methode, is te vinden in algoritme \ref{bisectie}.

\begin{algorithm}
\centering
\caption{Bisectie-methode}
    \label{bisectie}
\begin{algorithmic}
\State{queue = [[a,b)]}
\While{queue niet leeg}
\State{Haal eerste interval uit de queue}
\State{$Sturm_{links}$ = aantal tekenwisselingen in de Sturm-rij van A-linkergrens*I}
\State{$Sturm_{rechts}$ = aantal tekenwisselingen in de Sturm-rij van A-rechtergrens*I}
\State{\#eigenwaarden = $Sturm_{rechts}$ - $Sturm_{links}$}
\State{midden = 0.5*(rechtergrens - linkergrens)}

\If {\#eigenwaarden == 1}
	\If{midden $<$ tolerantie}
		\State{Voeg linkergrens+midden toe als eigenwaarde}
	\Else
    \State {Voeg [linkergrens,linkergrens+midden] en [linkergrens+midden,rechtergrens] vooraan bij in de queue}
    \EndIf
\ElsIf {\#eigenwaarden $>$ 1}
        \State {Voeg [linkergrens,linkergrens+midden] en [linkergrens+midden,rechtergrens] vooraan bij in de queue}
\EndIf
\EndWhile
\end{algorithmic}
\end{algorithm}

Als eerste voorbeeld nemen we de tridiagonale, symmetrische en re\"ele matrix
$$A = \begin{bmatrix} 
1 &1 &0 &0\\
1 &0 &1 &0 \\
0 &1 &2 &-1\\
0 &0 &1 &-1
\end{bmatrix}.$$
Als we hieruit de eigenwaarden willen berekenen die tussen 
