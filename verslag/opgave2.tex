De functie Householder\_explicit wordt weergegeven in onderstaande MATLAB-code. Deze methode genereert de matrix $Q$ door na de impliciete methode een nieuwe loop uit te voeren. Deze loop is gebaseerd op de functie om het product $Qx$ te berekenen, met voor $x$ telkens een eenheidsvector.
$\backslash$
Na het expliciet berekenen van de QR-factorisatie van A, wordt de oplossing van het stelsel $Ax=b$ bekomen door $y = Q^*b$ en $x= R\backslash y$ te berekenen.

\lstinputlisting[
  style      = Matlab-editor,
  basicstyle = \mlttfamily,
]{Householder_explicit.m}

De functies Householder\_implicit en Apply\_Q zijn Hier onder weergegeven in MATLAB-code.
Na het berekenen van L en R en L toe te passen op b (noem deze vector y), wordt de oplossing van het stelsel $Ax=b$ bekomen door $x= R\backslash y$ te berekenen. 

\lstinputlisting[
  style      = Matlab-editor,
  basicstyle = \mlttfamily,
]{Householder_implicit.m}

\lstinputlisting[
  style      = Matlab-editor,
  basicstyle = \mlttfamily,
]{Apply_Q.m}

De tijd om het stelsel $Ax=b$ op te lossen met de expliciete en de impliciete methode worden respectievelijk weergegeven in tabel \ref{snelheid_exp} en tabel \ref{snelheid_imp}. Hier zien we dat de tijd om het stelsel op te lossen een orde-grootte kleiner is met de impliciete methode dan met de expliciete methode voor een grotere waarde van $n$. Voor kleine $n$ is er geen verschil in uitvoeringstijd.

De ordegroottes van de relatieve fout $\lVert \delta x \rVert/\lVert x \rVert$ van de oplossing voor de expliciete en de impliciete methode worden respectievelijk weergegeven in tabel \ref{dx_exp} en tabel \ref{dx_imp}.
De ordegroottes van de verhouding van de norm van het residu op de norm van de b-vector $\lVert r \rVert/\lVert b \rVert$ van de oplossing voor de expliciete en de impliciete methode worden respectievelijk weergegeven in tabel \ref{rb_exp} en \ref{rb_imp}. Deze waarden vullen we nu in in de vergelijking
$$ \frac{\lVert \delta x \rVert}{\lVert x \rVert} \leq \kappa(A) \frac{\lVert r \rVert}{\lVert b \rVert}.$$
We zien hier dat het linkerlid van de ongelijkheid veel kleiner is dan het rechterlid naarmate $\kappa$ groter wordt. Voor $\kappa = 1$ zien we namelijk dat het linkerlid ongeveer gelijk is aan het rechterlid. Voor $\kappa = 10^8$ is er veel meer verschil tussen de twee leden. We zien zelfs dat wanneer $\kappa = 1$, de ongelijkheid niet meer helemaal klopt. Dit komt omdat de machine-precisie dan de beperkende factor wordt. We zien ook voor beide methodes dezelfde getallen, dus de mate van achterwaartse stabiliteit van de twee methodes is ongeveer dezelfde.

\begin{table}[H]
\begin{center}
\begin{tabular}{r|llc}
n $\backslash$ $\kappa$ & $1$ & $10^4$ & $10^8$ \\\hline
10 & 0,0044 & 0,0022 & 0,0025 \\
100 & 0,1883 & 0,1403 & 0,1306 \\
1000 & 28,2691 & 27,725 & 27,9583
\end{tabular}
\end{center}
\caption{De snelheid van de expliciete methode}
\label{snelheid_exp}
\end{table}

\begin{table}[H]
\begin{center}
\begin{tabular}{r|llc}
n $\backslash$ $\kappa$ & $1$ & $10^4$ & $10^8$ \\\hline
10 & 0,0050 & 0,0015 & 0,0013 \\
100 & 0,0115 & 0,0131 & 0,0163 \\
1000 & 7,9605 & 7,7967 & 7,8586
\end{tabular}
\end{center}
\caption{De snelheid van de impliciete methode}
\label{snelheid_imp}
\end{table}

\begin{table}[H]
\begin{center}
\begin{tabular}{r|llc}
n $\backslash$ $\kappa$ & $1$ & $10^4$ & $10^8$ \\\hline
10 & $10^{-16}$ & $10^{-13}$ & $10^{-9}$ \\
100 & $10^{-15}$ & $10^{-13}$ & $10^{-9}$ \\
1000 & $10^{-14}$ & $10^{-13}$ & $10^{-9}$
\end{tabular}
\end{center}
\caption{De ordegrootte van $\lVert \delta x \rVert/\lVert x \rVert$ van de expliciete methode}
\label{dx_exp}
\end{table}

\begin{table}[H]
\begin{center}
\begin{tabular}{r|llc}
n $\backslash$ $\kappa$ & $1$ & $10^4$ & $10^8$ \\\hline
10 & $10^{-16}$ & $10^{-13}$ & $10^{-9}$ \\
100 & $10^{-15}$ & $10^{-13}$ & $10^{-9}$ \\
1000 & $10^{-15}$ & $10^{-13}$ & $10^{-9}$
\end{tabular}
\end{center}
\caption{De ordegrootte van $\lVert \delta x \rVert/\lVert x \rVert$ van de impliciete methode}
\label{dx_imp}
\end{table}

\begin{table}[H]
\begin{center}
\begin{tabular}{r|llc}
n $\backslash$ $\kappa$ & $1$ & $10^4$ & $10^8$ \\\hline
10 & $10^{-16}$ & $10^{-13}$ & $10^{-9}$ \\
100 & $10^{-15}$ & $10^{-13}$ & $10^{-9}$ \\
1000 & $10^{-14}$ & $10^{-13}$ & $10^{-10}$
\end{tabular}
\end{center}
\caption{De ordegrootte van $\lVert r \rVert/\lVert b \rVert$ van de expliciete methode}
\label{rb_exp}
\end{table}

\begin{table}[H]
\begin{center}
\begin{tabular}{r|llc}
n $\backslash$ $\kappa$ & $1$ & $10^4$ & $10^8$ \\\hline
10 & $10^{-16}$ & $10^{-13}$ & $10^{-9}$ \\
100 & $10^{-15}$ & $10^{-13}$ & $10^{-10}$ \\
1000 & $10^{-14}$ & $10^{-13}$ & $10^{-10}$
\end{tabular}
\end{center}
\caption{De ordegrootte van $\lVert r \rVert/\lVert b \rVert$ van de impliciete methode}
\label{rb_imp}
\end{table}